\part{树}
\section{二叉树的中序遍历}

\textbf{题目描述:}

给定一个二叉树的根节点 root ,返回 它的 中序 遍历 。

\textbf{解法一:}
\begin{lstlisting}
    const inorderTraversal = (root) => {
        const result = [];
        const goThrough = (node) => {
            if (!node) {
                return;
            }
            goThrough(node.left);
            result.push(node.val);
            goThrough(node.right);
        }
        goThrough(root)

        return result;
    }
\end{lstlisting}

\textbf{解法二}
\begin{lstlisting}
    const inorderTraversal = (root) => {
        const result = [];
        const stack = [];
        while (root || stack.length) {
            while (root) {
                stack.push(root);
                root = root.left;
            }
            root = stack.pop();
            result.push(root.val);
            root = root.right;
        }
        
        return result;
    };
\end{lstlisting}

\section{二叉树的最大深度}

\textbf{题目描述:}
给定一个二叉树 root ,返回其最大深度。

二叉树的 最大深度 是指从根节点到最远叶子节点的最长路径上的节点数。

\textbf{解法一:}
\begin{lstlisting}
    const maxDepth = (root) => {
        let goThrough = (node) => {
            if (!node) {
                return 0;
            } else {
                const leftHeight = goThrough(node.left);
                const rightHeight = goThrough(node.right);
                return Math.max(leftHeight, rightHeight) + 1;
            }
        }
        
        return goThrough(root);
    }
\end{lstlisting}

\textbf{解法二:}
\begin{lstlisting}
    const maxDepth = (root) => {
        if (!root) {
            return 0;
        }
        
        const queue = [root];
        let ans = 0;
        
        while (queue.length > 0) {
            const size = queue.length;
            
            for (let i = 0; i < size; i++) {
                const node = queue.shift();
                
                if (node.left) {
                    queue.push(node.left);
                }
                if (node.right) {
                    queue.push(node.right);
                }
            }
            ans++;
        }
        
        return ans;
    };
\end{lstlisting}

\section{翻转二叉树}

\textbf{题目描述:}
给你一棵二叉树的根节点 root ,翻转这棵二叉树,并返回其根节点。

\textbf{解法一:}
\begin{lstlisting}
    const invertTree = (root) => {
        const goThrough = (node) => {
            if (!node) {
                return node;
            }
            let leftNode = node.left;
            let rightNode = node.right;
            node.left = rightNode;
            node.right = leftNode;
            if (leftNode) {
                goThrough(leftNode);
            }
            if (rightNode) {
                goThrough(rightNode);
            }
        }
        goThrough(root);

        return root;
    };
\end{lstlisting}

\section{对称二叉树}

\textbf{题目描述:}
给你一个二叉树的根节点 root , 检查它是否轴对称。

\textbf{解法一:}
\begin{lstlisting}
    const isSymmetric = (root) => {
        if (!root) {
            return true;
        }
        return recur(root.left, root.right);
        function recur(left, right) {
            if (!left && !right) {
                return true;
            }
            if (!left || !right || left.val !== right.val) {
                return false;
            }
            return recur(left.left, right.right) && recur(left.right, right.left);
        }
    }
\end{lstlisting}

\textbf{解法二:}
\begin{lstlisting}
    const isSymmetric = (root) => {
        if (!root) {
            return true;
        }
        
        const queue = [root.left, root.right];
        while (queue.length) {
            const left = queue.shift();
            const right = queue.shift();
            if (!left && !right) {
                continue;
            }
            if (!left || !right || left.val !== right.val) {
                return false;
            }
            queue.push(left.left, right.right);
            queue.push(left.right, right.left);
        }
        
        return true;
    }
\end{lstlisting}

\section{二叉树的直径}

\textbf{题目描述:}
给你一棵二叉树的根节点,返回该树的 直径 。

二叉树的 直径 是指树中任意两个节点之间最长路径的 长度 。这条路径可能经过也可能不经过根节点 root 。

两节点之间路径的 长度 由它们之间边数表示。

\textbf{解法一:}
\begin{lstlisting}
    const diameterOfBinaryTree = (root) => {
        let result = 0;
        const getDepth = (node) => {
            if (!node) {
                return 0;
            }
            let leftDepth = getDepth(node.left);
            let rightDepth = getDepth(node.right);
            result = Math.max(result, leftDepth + rightDepth);
            return Math.max(leftDepth, rightDepth) + 1;
        };
        getDepth(root);
        
        return result;
    };
\end{lstlisting}

\section{二叉树的层序遍历}

\textbf{题目描述:}
给你二叉树的根节点 root ,返回其节点值的 层序遍历 。 (即逐层地,从左到右访问所有节点)。

\textbf{解法一:}
\begin{lstlisting}
    levelOrder = function(root) {
        let result = [];
    
        if (!root) {
            return result;
        }
        const queue = [root]
        while (queue.length) {
            const nodesAtLevel = queue.length;
            result.push([]);
            for (let i = 0; i < nodesAtLevel; i++) {
                const node = queue.shift();
                result[result.length -1].push(node.val);
                if (node.left) {
                    queue.push(node.left);
                }
                if (node.right) {
                    queue.push(node.right);
                }
            }
        }
    
        return result;
    };
\end{lstlisting}

\section{将有序数组转换为二叉搜索树}

\textbf{题目描述:}

给你一个整数数组 nums ,其中元素已经按 升序 排列,请你将其转换为一棵 平衡 二叉搜索树。

\textbf{解法一:}

\begin{lstlisting}
    const sortedArrayToBST = (nums) => {
        const assistFunc = (nums, leftIndex, rightIndex) => {
            if (leftIndex > rightIndex) {
                return null;
            }
            const midIndex = Math.floor((leftIndex + rightIndex) / 2);
            const root = new TreeNode(nums[midIndex], null, null);
            root.left = assistFunc(nums, leftIndex, midIndex -1);
            root.right = assistFunc(nums, midIndex + 1, rightIndex);

            return root;
        }
    
        return assistFunc(nums, 0, nums.length -1);
    };
\end{lstlisting}

\section{验证二叉搜索树}

\textbf{题目描述:}
给你一个二叉树的根节点 root ,判断其是否是一个有效的二叉搜索树。

有效 二叉搜索树定义如下:

节点的左子树只包含 严格小于 当前节点的数。

节点的右子树只包含 严格大于 当前节点的数。

所有左子树和右子树自身必须也是二叉搜索树。

\textbf{解法一:}
\begin{lstlisting}
    const isValidBST = (root) => {
        const check = (node, min, max) => {
            if (!node) {
                return true;
            }
    
            if (min >= node.val || max <= node.val) {
                return false;
            }
    
            return check(node.left, min, node.val) && check(node.right, node.val, max);
        }
        
        return check(root, -Infinity, Infinity);
    };
\end{lstlisting}

\section{二叉搜索树中第 K 小的元素}

\textbf{题目描述:}
给定一个二叉搜索树的根节点 root ,和一个整数 k ,请你设计一个算法查找其中第 k 小的元素(从 1 开始计数)。

\textbf{解法一:}
\begin{lstlisting}
    const kthSmallest = (root, k) => {
        let vals = [];
        const collectVals = (node) => {
            if (!node) {
                return
            }
            vals.push(node.val);
            collectVals(node.left);
            collectVals(node.right);
        }
    
        collectVals(root);
        
        return vals.sort((a, b) => a - b)[k -1];
    };
\end{lstlisting}

\textbf{解法二:}

\begin{lstlisting}
    const kthSmallest = (root, k) => {
        const stack = [];
    
        while (root !== null || stack.length) {
            while (root !== null) {
                stack.push(root);
                root = root.left
            }
            root = stack.pop();
            k--;
            if (k === 0) {
                break;
            }
            root = root.right;
        }
    
        return root.val;
    };
\end{lstlisting}

\section{二叉树的右视图}

\textbf{题目描述:}
给定一个二叉树的 根节点 root,想象自己站在它的右侧,按照从顶部到底部的顺序,返回从右侧所能看到的节点值。

\textbf{解法一:}

\begin{lstlisting}
    const rightSideView = (root) => {
        let result = [];
        let queue = [];
        if (!root) {
            return result;
        }
        queue=[root];
        
        while(queue.length) {
            const items = queue.length;//This is key item.
            for (let i = 0; i < items; i++) {
                let node = queue.shift();
                if (i == items -1) {
                    result.push(node.val);
                }
                if (node.left) {
                    queue.push(node.left);
                }
                if (node.right) {
                    queue.push(node.right);
                }
            }
        }
        
        return result;
    };
\end{lstlisting}

\section{二叉树展开为链表}

\textbf{题目描述:}
给你二叉树的根结点 root ,请你将它展开为一个单链表:

展开后的单链表应该同样使用 TreeNode ,其中 right 子指针指向链表中下一个结点,而左子指针始终为 null 。
展开后的单链表应该与二叉树 先序遍历 顺序相同。

\textbf{解法一:}
\begin{lstlisting}
    const flatten = (root) => {
        const preorderTraversal = (root, list) => {
            if (root) {
              list.push(root);
              preorderTraversal(root.left, list);
              preorderTraversal(root.right, list);
            }
        }
    
        const list = [];
        preorderTraversal(root, list);
        const totalNodes = list.length;
    
        for (let i = 1; i < totalNodes; i++) {
            list[i -1].left = null;
            list[i -1].right = list[i];
        }
    };
\end{lstlisting}

\textbf{解法二:}

\begin{lstlisting}
    const flatten2 = (root) => {
        if (root === null) {
            return;
        }
        const stack = [];
        stack.push(root);
        let prev = null;
        while (stack.length) {
            const curr = stack.pop();
            if (prev !== null) {
                prev.left = null;
                prev.right = curr;
            }
            const left = curr.left, right = curr.right;
            if (right !== null) {
                stack.push(right);
            }
            if (left !== null) {
                stack.push(left);
            }
            prev = curr;
        }
    };
\end{lstlisting}

\section{从前序与中序遍历序列构造二叉树}

\textbf{题目描述:}

给定两个整数数组 preorder 和 inorder ,其中 preorder 是二叉树的先序遍历, inorder 是同一棵树的中序遍历,请构造二叉树并返回其根节点。

概念:先序遍历是先遍历根节点,然后遍历左子树,最后遍历右子树。中序遍历是遍历左子树,然后遍历根节点,最后遍历右子树。后序遍历是遍历左子树,然后遍历右子树,最后遍历根节点。

\textbf{解法一:}
\begin{lstlisting}
    const buildTree = function(preorder, inorder) {
        const inorderMap = new Map();

        for (let i = 0; i < inorder.length; i++) {
            inorderMap.set(inorder[i], i);
        }

        const build = (preStart, preEnd, inorderStart, inorderEnd) => {
            if (preStart > preEnd) {
                return null;
            }

            const rootVal = preorder[preStart];
            const rootIndex = inorderMap.get(rootVal);
            const root = new TreeNode(rootVal);
            const leftSize = rootIndex - inorderStart;

            // left subtree's preorder range: preStart+1 to preStart+leftSize
            // left subtree's inorder range: inStart to rootIndex-1
            root.left = build(
                preStart + 1,
                preStart + leftSize,
                inorderStart,
                rootIndex -1);

            // right subtree's preorder range: preStart+leftSize+1 to preEnd
            // right subtree's inorder range: rootIndex+1 to inEnd
            root.right = build(
                preStart + leftSize + 1, 
                preEnd, 
                rootIndex + 1, 
                inorderEnd);

            return root;
        }
        
        return build(0, preorder.length - 1, 0, inorder.length - 1);
    };
\end{lstlisting}

\textbf{解法二:}
\begin{lstlisting}
    const buildTree2 = (preorder, inorder) => {
        const inorderMap = new Map();
        for (let i = 0; i < inorder.length; i++) {
            inorderMap.set(inorder[i], i);
        }
        let preRootIndex = 0;

        const build = (inorderStar, inorderEnd) => {
            if (inorderStar > inorderEnd) {
                return null;
            }
            const node = new TreeNode(preorder[preRootIndex]);
            const inorderRootIndex = inorderMap.get(preorder[preRootIndex]);
            preRootIndex++
            node.left = build(inorderStar, inorderRootIndex - 1);
            node.right = build(inorderRootIndex + 1, inorderEnd);
            return node;
        }

        return build(0, inorder.length -1)
    };
\end{lstlisting}

\textbf{解法三:}
\begin{lstlisting}
    const buildTree = function(preorder, inorder) {
        if (!preorder || preorder.length === 0) {
            return null;
        }

        const stack = [];
        const root = new TreeNode(preorder[0]);
        stack.push(root);

        let inorderIndex = 0;

        for (let i = 1; i < preorder.length; i++) {
            let node = stack[stack.length - 1];

            if (node.val !== inorder[inorderIndex]) {
                node.left = new TreeNode(preorder[i]);
                stack.push(node.left);
            } else {
                // node.val === inorder[inorderIndex] means that the left subtree is complete. 
                while (stack.length > 0 && stack[stack.length - 1].val === inorder[inorderIndex]) {
                    node = stack.pop();
                    inorderIndex++;
                }
                node.right = new TreeNode(preorder[i]);
                stack.push(node.right);
            }
        }

        return root;
    };
\end{lstlisting}

\section{路径总和 III}

\textbf{题目描述:}
给定一个二叉树的根节点 root ,和一个整数 targetSum ,求该二叉树里节点值之和等于 targetSum 的 路径 的数目。

路径 不需要从根节点开始,也不需要在叶子节点结束,但是路径方向必须是向下的(只能从父节点到子节点)。

\textbf{解法一:}
\begin{lstlisting}
    const pathSum = function(root, targetSum) {
        if (root === null) {
            return 0;
        }
    
        let result = rootSum(root, targetSum); // count the root itself
        result += pathSum(root.left, targetSum); // count the left subtree
        result += pathSum(root.right, targetSum);
    
        return result;
    };
    
    const rootSum = (root, targetSum) => {
        let result = 0;
    
        if (root === null) {
            return 0;
        }
    
        const val = root.val;
    
        if (val === targetSum) {
            result ++;
        }
    
        result += rootSum(root.left, targetSum - val);
        result += rootSum(root.right, targetSum - val);
    
        return result;
    }
\end{lstlisting}

\textbf{解法二:}
\begin{lstlisting}
    const pathSum = (root, targetSum) => {
        const prefix = new Map();
        prefix.set(0, 1);
        return dfs(root, prefix, 0, targetSum);
    }

    const dfs = (root, prefix, sum, targetSum) => {
        if (!root) {
            return 0;
        }
        sum += root.val;
        let res = prefix.get(sum - targetSum) || 0;
        prefix.set(sum, (prefix.get(sum) || 0) + 1);
        res += dfs(root.left, prefix, sum, targetSum);
        res += dfs(root.right, prefix, sum, targetSum);
        prefix.set(sum, prefix.get(sum) - 1);
        return res;
    }
    
\end{lstlisting}


\section{二叉树的最近公共祖先}

\textbf{题目描述:}
给定一个二叉树, 找到该树中两个指定节点的最近公共祖先。

百度百科中最近公共祖先的定义为:“对于有根树 T 的两个节点 p、q,最近公共祖先表示为一个节点 x,满足 x 是 p、q 的祖先且 x 的深度尽可能大(一个节点也可以是它自己的祖先)。”

\textbf{解法一:}
\begin{lstlisting}
    const lowestCommonAncestor = (root, p, q) => {
        
        if (!root || root === p || root === q) {
            return root;
        }
        const leftChild = lowestCommonAncestor(root.left, p, q);
        const rightChild = lowestCommonAncestor(root.right, p, q);

        if (leftChild && rightChild) {
            return root;
        }

        return leftChild || rightChild;
    }

\end{lstlisting}

\textbf{解法二:}
\begin{lstlisting}
    let parentMap = new Map();
    let visitedVal = new Set();

    const dfs = (root) => {
        if (root.left !== null) {
            parentMap.set(root.left.val, root);
            dfs(root.left);
        }

        if (root.right !== null) {
            parentMap.set(root.right.val, root);
            dfs(root.right);
        }
    }
    dfs(root);

    while (p) {
        visitedVal.add(p.val);
        p = parentMap.get(p.val);
    }

    while (q) {
        if (visitedVal.has(q.val)) {
            return q;
        }
        q = parentMap.get(q.val);
    }

    return null;
\end{lstlisting}

\textbf{解法三:}
\begin{lstlisting}
    const lowestCommonAncestor = (root, p, q) => {
        const findPath = (node, target, path) => {
            if (!node) return false;
            
            path.push(node);
            
            if (node === target) {
                return true;
            }
            
            if (findPath(node.left, target, path)) {
                return true;
            }
            
            if (findPath(node.right, target, path)) {
                return true;
            }
            
            path.pop(); //This is very important. backtrack.
            return false;
        };
        
        const pathP = [];
        const pathQ = [];
        
        findPath(root, p, pathP);
        findPath(root, q, pathQ);
        
        let lca = null;
        const len = Math.min(pathP.length, pathQ.length);
        
        for (let i = 0; i < len; i++) {
            if (pathP[i] === pathQ[i]) {
                lca = pathP[i];
            } else {
                break;
            }
        }
        
        return lca;
    }
\end{lstlisting}


\section{二叉树中的最大路径和}
\textbf{题目描述:}

二叉树中的 路径 被定义为一条节点序列,序列中每对相邻节点之间都存在一条边。同一个节点在一条路径序列中 至多出现一次 。该路径 至少包含一个 节点,且不一定经过根节点。
路径和 是路径中各节点值的总和。
给你一个二叉树的根节点 root ,返回其 最大路径和。

\textbf{解法一:}
\begin{lstlisting}
    const maxPathSum = (root) => {
        let maxSum = -Infinity;
    
        const dfs = (node) => {
            if (!node) {
                return 0;
            }
            const leftGain = Math.max(dfs(node.left), 0);
            const rightGain = Math.max(dfs(node.right), 0);
            const currentGain = node.val + leftGain + rightGain;
            maxSum = Math.max(maxSum, currentGain);
    
            return Math.max(leftGain, rightGain) + node.val;
        }
    
        dfs(root);
    
        return maxSum
    };
\end{lstlisting}