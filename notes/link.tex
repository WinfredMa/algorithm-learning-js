\part{链表}

\section{两数相加}

\textbf{题目描述:}

给你两个 非空 的链表,表示两个非负的整数。它们每位数字都是按照 逆序 的方式存储的,并且每个节点只能存储 一位 数字。

请你将两个数相加,并以相同形式返回一个表示和的链表。

你可以假设除了数字 0 之外,这两个数都不会以 0 开头。

\textbf{解法一:}

\begin{lstlisting}
    const addTwoNumbers = (l1, l2) => {
        const dummyHead = new ListNode(0);
        let curr = dummyHead;
        let carry = 0;
        while (l1 || l2 || carry) {
            const val1 = l1 ? l1.val : 0;
            const val2 = l2 ? l2.val : 0;
            const sum = val1 + val2 + carry;
            carry = Math.floor(sum / 10);
            curr.next = new ListNode(sum % 10);
            curr = curr.next;
            l1 = l1 ? l1.next : null;
            l2 = l2 ? l2.next : null;
        }
        return dummyHead.next;
    };
\end{lstlisting}

\section{相交链表}
\textbf{题目描述:}

给你两个单链表的头节点 headA 和 headB ,请你找出并返回两个单链表相交的起始节点。如果两个链表不存在相交节点,返回 null 。

\textbf{解法一:}

\begin{lstlisting}
    const getIntersectionNode = (headA, headB) => {
        const visited = new Set();
        let tempNode = headA;
        while (tempNode) {
            visited.add(tempNode);
            tempNode = tempNode.next;
        }

        tempNode = headB;
        while (tempNode) {
            if (visited.has(tempNode)) {
                return tempNode;
            }
            tempNode = tempNode.next;
        }

        return null;
    };
\end{lstlisting}

\textbf{解法二:}

\begin{lstlisting}
    const getIntersectionNode2 = (headA, headB) =>{
        if (!headA || !headB) {
            return null;
        }
        
        let ptrA = headA;
        let ptrB = headB;
        
        while (ptrA !== ptrB) {
            ptrA = ptrA ? ptrA.next : headB;
            ptrB = ptrB ? ptrB.next : headA;
        }
        
        return ptrA;
    };
\end{lstlisting}

\section{反转链表}

\textbf{题目描述:}

给你单链表的头节点 head ,请你反转链表,并返回反转后的链表。

\textbf{解法一:}

\begin{lstlisting}
    const reverseList = (head) => {
        let prev = null;
        let curr = head;
        
        while(curr) {
            let next = curr.next;
            curr.next = prev;
            prev = curr;
            curr = next;
        }
        
        return prev;
    };
\end{lstlisting}

\section{回文链表}
\textbf{题目描述:}

给你一个单链表的头节点 head ,请你判断该链表是否为回文链表。如果是,返回 true ;否则,返回 false 。

\textbf{解法一:}

\begin{lstlisting}
    const isPalindrome = (head) => {
        let vals = [];
        while(head) {
            vals.push(head.val);
            head = head.next;
        }

        let left = 0, right = vals.length - 1;

        while (left < right) {
            if (vals[left] !== vals[right]) {
                return false;
            }
            left++;
            right--;
        }

        return true;
    };
\end{lstlisting}

\section{环形链表}

\textbf{题目描述:}

给你一个链表的头节点 head ,判断链表中是否有环。

如果链表中有某个节点,可以通过连续跟踪 next 指针再次到达,则链表中存在环。 为了表示给定链表中的环,评测系统内部使用整数 pos 来表示链表尾连接到链表中的位置(索引从 0 开始)。注意:pos 不作为参数进行传递 。仅仅是为了标识链表的实际情况。

如果链表中存在环 ,则返回 true 。 否则,返回 false 。

\textbf{解法一:}

\begin{lstlisting}
    const hasCycle = (head) => {
        let slowNode = head, fastNode = head;

        if (head === null || head.next === null) {
            return false;
        }

        while(fastNode && fastNode.next) {
            slowNode = slowNode.next;
            fastNode = fastNode.next.next;
            if (fastNode === slowNode) {
                return true;
            }
        }
        
        return false;
    }
\end{lstlisting}

\section{环形链表 II}

\textbf{题目描述:}

给定一个链表的头节点  head ,返回链表开始入环的第一个节点。 如果链表无环,则返回 null。

如果链表中有某个节点,可以通过连续跟踪 next 指针再次到达,则链表中存在环。 为了表示给定链表中的环,评测系统内部使用整数 pos 来表示链表尾连接到链表中的位置(索引从 0 开始)。如果 pos 是 -1,则在该链表中没有环。注意:pos 不作为参数进行传递,仅仅是为了标识链表的实际情况。

不允许修改 链表。

\textbf{解法一:}

\begin{lstlisting}
    const detectCycle = (head) => {
        let result = null,slowNode = head, fastNode = head;

        while (fastNode && fastNode.next) {
            slowNode = slowNode.next;
            fastNode = fastNode.next.next;

            if (fastNode === slowNode) {
                slowNode = head;
                while (slowNode !== fastNode) {
                    slowNode = slowNode.next;
                    fastNode = fastNode.next;
                }

                return slowNode;
            }
        }
        
        return result;
    };
\end{lstlisting}

\section{合并两个有序链表}

\textbf{题目描述:}

将两个升序链表合并为一个新的 升序 链表并返回。新链表是通过拼接给定的两个链表的所有节点组成的。

\textbf{解法一:}

\begin{lstlisting}
    const mergeTwoLists = (list1, list2) => {
        if (!list1 || !list2) {
            return list1 || list2;
        }
        let dummy = new ListNode(0, null);
        let current = dummy;
        while (list1 && list2) {
            if (list1.val > list2.val) {
                let temp = new ListNode(list2.val);
                current.next = temp;
                current = temp;
                list2 = list2.next;
            } else {
                let temp = new ListNode(list1.val);
                current.next = temp;
                current = temp;
                list1 = list1.next;
            }
        }
        if (list1) {
            current.next = list1;
        } else {
            current.next = list2;
        }
        
        return dummy.next
    };
\end{lstlisting}

\textbf{解法二:}

\begin{lstlisting}
    const mergeTwoLists2 = (list1, list2) => {
        if (list1 === null) {
            return list2;
        } else if (list2 === null) {
            return list1;
        } else if (list1.val < list2.val) {
            list1.next = mergeTwoLists2(list1.next, list2);
            return list1;
        } else {
            list2.next = mergeTwoLists2(list1, list2.next);
            return list2;
        }
    };
\end{lstlisting}

\textbf{解法三:}

\begin{lstlisting}
    const mergeTwoLists3 = (list1, list2) => {
        const prehead = new ListNode(-1);
        let prev = prehead;

        while (list1 && list2) {
            if (list1.val <= list2.val) {
                prev.next = list1;
                list1 = list1.next;
            } else {
                prev.next = list2;
                list2 = list2.next;
            }
            prev = prev.next;
        }
        prev.next = list1 === null ? list2 : list1;
        
        return prehead.next;
    };
\end{lstlisting}

\section{删除链表的倒数第 N 个结点}
\textbf{题目描述:}

给你一个链表,删除链表的倒数第 n 个结点,并且返回链表的头结点。

\textbf{解法一:}

\begin{lstlisting}
    const removeNthFromEnd = (head, n) => {
        let dummy = new ListNode(0, head), fastNode = head,slowNode = dummy;

        while (fastNode !== null) {
            if (n <=0 ) {
                slowNode = slowNode.next;
            }
            n--;
            fastNode = fastNode.next;
        }
        slowNode.next = slowNode.next.next;
        return dummy.next;
    };
\end{lstlisting}

\section{两两交换链表中的节点}
\textbf{题目描述:}

给你一个链表,两两交换其中相邻的节点,并返回交换后链表的头节点。你必须在不修改节点内部的值的情况下完成本题(即,只能进行节点交换)。

\textbf{解法一:}

\begin{lstlisting}
    const swapPairs = (head) => {
        let curr = head;
        let prev = null;
        let newhead = null;
        if (!curr?.next) {
            return head;
        }

        while (curr?.next) {
            let temp = curr.next.next;
            if (prev) {
                prev.next = curr.next;
            }
            if (!newhead) {
                newhead = curr.next;
            }
            curr.next.next = curr;
            curr.next = temp;
            prev = curr;
            curr = temp;
        }
        
        return newhead
    };
\end{lstlisting}

\textbf{解法二:}

\begin{lstlisting}
    const swapPairs2 = (head) => {
        if (head === null || head.next === null) {
            return head;
        }
        const newHead = head.next;
        head.next = swapPairs(newHead.next);
        newHead.next = head;
        return newHead;
    };
\end{lstlisting}

\section{K个一组翻转链表}
\textbf{题目描述:}

给你链表的头节点 head ,每 k 个节点一组进行翻转,请你返回修改后的链表。

k 是一个正整数,它的值小于或等于链表的长度。如果节点总数不是 k 的整数倍,那么请将最后剩余的节点保持原有顺序。

你不能只是单纯的改变节点内部的值,而是需要实际进行节点交换。

\textbf{解法一:}

\begin{lstlisting}
    const reverseKGroup = (head, k) => {
        const reverseSubLink = (head, tail) => {
            let previousNext = tail.next;
            let p = head;
            while (previousNext !== tail) {
                const next = p.next;
                p.next = previousNext;
                previousNext = p;
                p = next;
            }
            return [tail, head];
        }
        const dummy = new ListNode(0, head);
        let pre = dummy;

        while (head) {
            let tail = pre;
            for (let i = 0; i < k; i++) {
                tail = tail.next;
                if (!tail) {
                    return dummy.next;
                }
            }

            [head, tail] = reverseSubLink(head, tail);
            pre.next = head;
            pre = tail;
            head = tail.next;
        }
        return dummy.next;
    };
\end{lstlisting}

\section{随机链表的复制}
\textbf{题目描述:}

给你一个长度为 n 的链表,每个节点包含一个额外增加的随机指针 random ,该指针可以指向链表中的任何节点或空节点。

构造这个链表的 深拷贝。 深拷贝应该正好由 n 个 全新 节点组成,其中每个新节点的值都设为其对应的原节点的值。新节点的 next 指针和 random 指针也都应指向复制链表中的新节点,并使原链表和复制链表中的这些指针能够表示相同的链表状态。复制链表中的指针都不应指向原链表中的节点 。

例如,如果原链表中有 X 和 Y 两个节点,其中 X.random --> Y 。那么在复制链表中对应的两个节点 x 和 y ,同样有 x.random --> y 。

返回复制链表的头节点。

\textbf{解法一:}

\begin{lstlisting}
    const copyRandomList = (head) => {
        if (!head) {
            return null;
        }
        
        const map = new Map();
        let current = head;
        
        while (current) {
            map.set(current, new Node(current.val));
            current = current.next;
        }
        
        current = head;
        while (current) {
            // Key logics
            const newNode = map.get(current);
            newNode.next = map.get(current.next) || null;
            newNode.random = map.get(current.random) || null;
            current = current.next;
        }
        
        return map.get(head);
    };
\end{lstlisting}

\textcolor{red}{*体会map在该场景下的用法}

\section{排序链表}
\textbf{题目描述:}

给你链表的头结点 head ,请将其按 升序 排列并返回 排序后的链表 。

\textbf{解法一:}

\begin{lstlisting}
    const sortList = (head) => {
        let prev = null, slow = head, fast = head;
        const merge = (link1, link2) => {
            let dummy = new ListNode(0, null);
            let curr = dummy;
            while (link1 && link2) {
                if (link1.val >= link2.val) {
                    curr.next = link2;
                    link2 = link2.next;
                } else {
                    curr.next = link1;
                    link1 = link1.next;
                }
                curr = curr.next;
            }

            curr.next = link1 || link2;

            return dummy.next;
        }

        if (!head || !head.next) {
            return head;
        }

        while (fast && fast.next) {
            prev = slow;
            slow = slow.next;
            fast = fast.next.next;
        }

        prev.next = null;

        return merge(sortList(head), sortList(slow))
    };
\end{lstlisting}

\textbf{解法二:}

\begin{lstlisting}
    const sortList2 = (head) => {
        const mergeLink = (link1, link2) => {
            const dummy = new ListNode(0);
            let current = dummy

            while(link1 !== null && link2 !== null) {
                if (link1.val <= link2.val) {
                    current.next = link1;
                    link1 = link1.next;
                } else {
                    current.next = link2;
                    link2 = link2.next;
                }
                current = current.next;
            }
            current.next = link1 || link2;

            return dummy.next;
        }

        const toSortList = (head, tail) => {
            if (head === null) {
                return head;
            }
            if (head.next === tail) {
                head.next = null;
                return head;
            }

            let slow = head, fast = head;
            while (fast !== tail) {
                slow = slow.next;
                fast = fast.next;

                if (fast !== tail) {
                    fast = fast.next;
                }
            }
            const mid = slow;
            return mergeLink(toSortList(head, mid), toSortList(mid, tail));
        };
        
        return toSortList(head, null);
    }
\end{lstlisting}

\section{合并K个升序链表}
\textbf{题目描述:}

给你一个链表数组,每个链表都已经按升序排列。

请你将所有链表合并到一个升序链表中,返回合并后的链表。

\textbf{解法一:}

\begin{lstlisting}
    const mergeKLists = (lists) => {
        let dummy = new ListNode(0)
        let link1 = lists[0];
        let curr = dummy;
        if (lists.length === 1) {
            return lists[0];
        }
        
        for (let i = 1; i < lists.length; i++) {
            let link2 = lists[i];
            while (link1 && link2) {
                if (link1.val > link2.val) {
                    curr.next = link2;
                    link2 = link2.next;
                } else {
                    curr.next = link1;
                    link1 = link1.next;
                }
                curr = curr.next;
            }
            curr.next = link1 || link2;

            link1 = dummy.next;
            curr = dummy;
        }

        return dummy.next;
    };
\end{lstlisting}

\section{LRU 缓存}
\textbf{题目描述:}

请你设计并实现一个满足 LRU (最近最少使用) 缓存 约束的数据结构。实现 LRUCache 类:

\textbf{解法一:}

\begin{lstlisting}
    const LRUCache = (capacity) => {
        this.capacity = capacity;
        this.map = new Map();
    };

    LRUCache.prototype.get = (key) => {
        if (this.map.has(key)) {
            const val = this.map.get(key);
            this.map.delete(key);
            this.map.set(key, val);
            return val
        }
        return -1;
    };

    LRUCache.prototype.put = (key, value) => {
        const item = this.map.get(key);
        if (item) {
            this.map.delete(key);
        } else {
            // this.map.keys().next().value
            if (this.capacity === this.map.size) {
                this.map.delete(this.map.keys().next().value);
            }
        }
        this.map.set(key, value);
    };
\end{lstlisting}

\textcolor{red}{*体会map在该场景下的用法}