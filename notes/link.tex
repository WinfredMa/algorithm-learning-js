\part{链表}

\section{两数相加}

\textbf{题目描述:}

给你两个 非空 的链表,表示两个非负的整数。它们每位数字都是按照 逆序 的方式存储的,并且每个节点只能存储 一位 数字。

请你将两个数相加,并以相同形式返回一个表示和的链表。

你可以假设除了数字 0 之外,这两个数都不会以 0 开头。

\textbf{解法一:}

\begin{lstlisting}
    const addTwoNumbers = (l1, l2) => {
        const dummyHead = new ListNode(0);
        let curr = dummyHead;
        let carry = 0;
        while (l1 || l2 || carry) {
            const val1 = l1 ? l1.val : 0;
            const val2 = l2 ? l2.val : 0;
            const sum = val1 + val2 + carry;
            carry = Math.floor(sum / 10);
            curr.next = new ListNode(sum % 10);
            curr = curr.next;
            l1 = l1 ? l1.next : null;
            l2 = l2 ? l2.next : null;
        }
        return dummyHead.next;
    };
\end{lstlisting}

\section{相交链表}
\textbf{题目描述:}

给你两个单链表的头节点 headA 和 headB ,请你找出并返回两个单链表相交的起始节点。如果两个链表不存在相交节点,返回 null 。

\textbf{解法一:}

\begin{lstlisting}
    const getIntersectionNode = (headA, headB) => {
        const visited = new Set();
        let tempNode = headA;
        while (tempNode) {
            visited.add(tempNode);
            tempNode = tempNode.next;
        }

        tempNode = headB;
        while (tempNode) {
            if (visited.has(tempNode)) {
                return tempNode;
            }
            tempNode = tempNode.next;
        }

        return null;
    };
\end{lstlisting}

\textbf{解法二:}

\begin{lstlisting}
    const getIntersectionNode2 = (headA, headB) =>{
        if (!headA || !headB) {
            return null;
        }
        
        let ptrA = headA;
        let ptrB = headB;
        
        while (ptrA !== ptrB) {
            ptrA = ptrA ? ptrA.next : headB;
            ptrB = ptrB ? ptrB.next : headA;
        }
        
        return ptrA;
    };
\end{lstlisting}

\section{反转链表}

\textbf{题目描述:}

给你单链表的头节点 head ,请你反转链表,并返回反转后的链表。

\textbf{解法一:}

\begin{lstlisting}
    const reverseList = (head) => {
        let prev = null;
        let curr = head;
        
        while(curr) {
            let next = curr.next;
            curr.next = prev;
            prev = curr;
            curr = next;
        }
        
        return prev;
    };
\end{lstlisting}

\section{回文链表}
\textbf{题目描述:}

给你一个单链表的头节点 head ,请你判断该链表是否为回文链表。如果是,返回 true ;否则,返回 false 。

\textbf{解法一:}

\begin{lstlisting}
    const isPalindrome = (head) => {
        let vals = [];
        while(head) {
            vals.push(head.val);
            head = head.next;
        }

        let left = 0, right = vals.length - 1;

        while (left < right) {
            if (vals[left] !== vals[right]) {
                return false;
            }
            left++;
            right--;
        }

        return true;
    };
\end{lstlisting}

\section{环形链表}

\textbf{题目描述:}

给你一个链表的头节点 head ,判断链表中是否有环。

如果链表中有某个节点,可以通过连续跟踪 next 指针再次到达,则链表中存在环。 为了表示给定链表中的环,评测系统内部使用整数 pos 来表示链表尾连接到链表中的位置(索引从 0 开始)。注意:pos 不作为参数进行传递 。仅仅是为了标识链表的实际情况。

如果链表中存在环 ,则返回 true 。 否则,返回 false 。

\textbf{解法一:}

\begin{lstlisting}
    const hasCycle = (head) => {
        let slowNode = head, fastNode = head;

        if (head === null || head.next === null) {
            return false;
        }

        while(fastNode && fastNode.next) {
            slowNode = slowNode.next;
            fastNode = fastNode.next.next;
            if (fastNode === slowNode) {
                return true;
            }
        }
        
        return false;
    }
\end{lstlisting}

\section{环形链表 II}

\textbf{题目描述:}

给定一个链表的头节点  head ,返回链表开始入环的第一个节点。 如果链表无环,则返回 null。

如果链表中有某个节点,可以通过连续跟踪 next 指针再次到达,则链表中存在环。 为了表示给定链表中的环,评测系统内部使用整数 pos 来表示链表尾连接到链表中的位置(索引从 0 开始)。如果 pos 是 -1,则在该链表中没有环。注意:pos 不作为参数进行传递,仅仅是为了标识链表的实际情况。

不允许修改 链表。

\textbf{解法一:}

\begin{lstlisting}
    const detectCycle = (head) => {
        let result = null,slowNode = head, fastNode = head;

        while (fastNode && fastNode.next) {
            slowNode = slowNode.next;
            fastNode = fastNode.next.next;

            if (fastNode === slowNode) {
                slowNode = head;
                while (slowNode !== fastNode) {
                    slowNode = slowNode.next;
                    fastNode = fastNode.next;
                }

                return slowNode;
            }
        }
        
        return result;
    };
\end{lstlisting}

\section{合并两个有序链表}

\textbf{题目描述:}

将两个升序链表合并为一个新的 升序 链表并返回。新链表是通过拼接给定的两个链表的所有节点组成的。

\textbf{解法一:}

\begin{lstlisting}
    const mergeTwoLists = (list1, list2) => {
        if (!list1 || !list2) {
            return list1 || list2;
        }
        let dummy = new ListNode(0, null);
        let current = dummy;
        while (list1 && list2) {
            if (list1.val > list2.val) {
                let temp = new ListNode(list2.val);
                current.next = temp;
                current = temp;
                list2 = list2.next;
            } else {
                let temp = new ListNode(list1.val);
                current.next = temp;
                current = temp;
                list1 = list1.next;
            }
        }
        if (list1) {
            current.next = list1;
        } else {
            current.next = list2;
        }
        
        return dummy.next
    };
\end{lstlisting}

\textbf{解法二:}

\begin{lstlisting}
    const mergeTwoLists2 = (list1, list2) => {
        if (list1 === null) {
            return list2;
        } else if (list2 === null) {
            return list1;
        } else if (list1.val < list2.val) {
            list1.next = mergeTwoLists2(list1.next, list2);
            return list1;
        } else {
            list2.next = mergeTwoLists2(list1, list2.next);
            return list2;
        }
    };
\end{lstlisting}

\textbf{解法三:}

\begin{lstlisting}
    const mergeTwoLists3 = (list1, list2) => {
        const prehead = new ListNode(-1);
        let prev = prehead;

        while (list1 && list2) {
            if (list1.val <= list2.val) {
                prev.next = list1;
                list1 = list1.next;
            } else {
                prev.next = list2;
                list2 = list2.next;
            }
            prev = prev.next;
        }
        prev.next = list1 === null ? list2 : list1;
        
        return prehead.next;
    };
\end{lstlisting}

\section{删除链表的倒数第 N 个结点}
\textbf{题目描述:}

给你一个链表,删除链表的倒数第 n 个结点,并且返回链表的头结点。

\textbf{解法一:}

\begin{lstlisting}
    const removeNthFromEnd = (head, n) => {
        let dummy = new ListNode(0, head), fastNode = head,slowNode = dummy;

        while (fastNode !== null) {
            if (n <=0 ) {
                slowNode = slowNode.next;
            }
            n--;
            fastNode = fastNode.next;
        }
        slowNode.next = slowNode.next.next;
        return dummy.next;
    };
\end{lstlisting}

\section{两两交换链表中的节点}
\textbf{题目描述:}

给你一个链表,两两交换其中相邻的节点,并返回交换后链表的头节点。你必须在不修改节点内部的值的情况下完成本题(即,只能进行节点交换)。

\textbf{解法一:}

\begin{lstlisting}
    const swapPairs = (head) => {
        let curr = head;
        let prev = null;
        let newhead = null;
        if (!curr?.next) {
            return head;
        }

        while (curr?.next) {
            let temp = curr.next.next;
            if (prev) {
                prev.next = curr.next;
            }
            if (!newhead) {
                newhead = curr.next;
            }
            curr.next.next = curr;
            curr.next = temp;
            prev = curr;
            curr = temp;
        }
        
        return newhead
    };
\end{lstlisting}

\textbf{解法二:}

\begin{lstlisting}
    const swapPairs2 = (head) => {
        if (head === null || head.next === null) {
            return head;
        }
        const newHead = head.next;
        head.next = swapPairs(newHead.next);
        newHead.next = head;
        return newHead;
    };
\end{lstlisting}    